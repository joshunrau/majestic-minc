\documentclass{article}

\usepackage[T1]{fontenc}
\usepackage{txfonts}

% Little macro to make keywords obvious.
\newcommand{\kw}[1]{{\tt \mbox{#1}}}

\title{perturb\_surface\_normal}
\author{Robert D. Vincent \\
McConnell Brain Imaging Centre \\
Montreal Neurological Institute \\
Montreal, QC H3A 2B4 Canada \\~\\
}
\date{\today}
\begin{document}
\maketitle

\section{Introduction}
This is a simple tool to introduce a controlled amount of distortion
into a polygonal surface, intended for testing and evaluation purposes.

The program takes seven parameters:
\begin{enumerate}
\item An input file name.
  \item An x-coordinate of a point in the space of the surface.
  \item An y-coordinate of a point in the space of the surface.
  \item An z-coordinate of a point in the space of the surface.
  \item A FWHM (full width at half maximum) parameter.
  \item A weight or amplitude parameter (misleadingly called
    ``distance'' in the code).
  \item An output file name.
\end{enumerate}

\section{Algorithm}
The representation of a polygon in the tool includes a list of
points $\mathbf{P}$ and a parallel list of normal vectors $\mathbf{N}$.

The $x$, $y$ and $z$ parameters define a point $c$ that
is used as the centre of a 3-dimensional Gaussian function. The Gaussian
controls the amount of perturbation applied to a given point.

For every point $p$ in the list $\mathbf{P}$, the
program computes a distance $d$ as the Euclidean distance from $c$ to
$p$:
\begin{equation}
  d = \sqrt{(p_x - c_x)^2 + (p_y - c_y)^2 + (p_z - c_z)^2}
\end{equation}
Taking $n$ as the normal vector that corresponds to the point $p$,
and we compute a new point $p'$ as follows:
\begin{eqnarray}
  p'_x = p_x + n_x a e^{-\frac{4\ln(2) d^2}{w^2}}\\
  p'_y = p_y + n_y a e^{-\frac{4\ln(2) d^2}{w^2}}\\
  p'_z = p_z + n_z a e^{-\frac{4\ln(2) d^2}{w^2}}
\end{eqnarray}
where $a$ is the amplitude parameter and $w$ is the FWHM parameter. In the
resulting surface, each 
each point $p$ in $\mathbf{P}$ is replaced with its perturbed point $p'$.

\end{document}
