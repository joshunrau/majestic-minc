\documentstyle[12pt]{article}

% [e,d]= do_test(2, 8, 4, 1/(256*256));
% [e,d]= do_test(2, 8, 4, 1/(256));

\begin{document}

The automatic recovery of the global non-linear warping used to
register two data sets is based on the estimation of local
neighbourhood deformations.  The goal is to individually maximize the
local correlation at each point of the whole volume while maintaining
spatial continuity of the volumes and the deformation maps. This
estimation is the result of correlation of a feature derived from each
image volume: the 2nd derivative along the gradient direction of the
blurred intensity volume.

There are a number of factors that affect the result of the
correlation.  First is the estimation of the feature itself. Four
input volumes are used in the creation of the second derivative
feature volume (D2).  The first is the blurred intensity volume (3D
gaussian blurring with a known fwhm) and the three others are the
blurred intensity derivatives; partial derivatives along each of the
three axes yielding dx, dy and dz volumes.  The partial derivative
volumes are used to estimate the direction of the intensity derivative
$\vec{v}$ at the center of each voxel in the D2 volume.  For each
voxel point $p$, 4 samples are then taken from the blurred intensity
volume along a line parallel to $\vec{v}$ passing through $p$, spaced
at $fwhm/2$ and centered on $p$.  The second derivative for $p$ is
estimated using cubic interpolation through these four points.

%  for 
%            I = a1 + a2*xpos + a3*xpos^2 + a4*xpos^3:   
% 
% 
%       a1 =                f2;
%       a2 =  -1.0*f1/3.0 - f2/2.0 + f3     - f4/6.0;
%       a3 =       f1/2.0 - f2     + f3/2.0 ;
%       a4 =  -1.0*f1/6.0 + f2/2.0 - f3/2.0 + f4/6.0;
%       
%       xpos = 0.0;
%       
%       fp  = a2 + 2*a3*xpos + 3*a4*xpos*xpos; / * first derivative * /
%       fpp = 2*a3 + 6*a4*xpos;	           / * second derivative * /

Noise in the data will of course affect the estimation of the
derivatives. When storing the blurred data volume in non-float format
(i.e.  one byte or one short per voxel), there is a discretization
error in the intensity value stored at each voxel.  When stored as
bytes, this error is on the order of $max$/256, and when stored as
short integers, is $max$/65536.  The derivative is a measure of change
in intensity over a distance.  Therefore, the voxel size (i.e. spacing
between voxel centers) also plays a role in the estimation of the
derivative.   Smaller voxels accentuate the discritization error.

For the voxel sizes used in this study (approximately 1mm$^3$), the
discretization errors give rise to errors in the first dirivative of
less than 3\% of maximum for byte data and 2\% of maximum for short
data.  Errors in the second derivative for byte data are less than 12%
of maximum value, and are approximately 6% for short data.

\end{document}
