%
% mni_perllib.tex
%
% Main LaTeX source file for the Programmer's Reference to the MNI Perl
% Library.  Do NOT just run this file through LaTeX without any of the
% necessary pre-processing steps.  (Hint: use the Makefile, it takes
% care of everything for you.  Try "make ps", or "make a4_ps" if you use
% European-sized paper.)
%
% Or, you can just download one of the PostScript files from the BIC's
% ftp site: ftp://ftp.bic.mni.mcgill.ca/pub/perl
%

% TODO:
% 
% * figure out fancyheaders and use it

\documentclass{article}
\usepackage{fullpage}
\usepackage{times}

% stuff to keep the LaTeX generated by pod2latex happy
\def\C++{{\rm C\kern-.05em\raise.3ex\hbox{\footnotesize ++}}}
\def\underscore{\leavevmode\kern.04em\vbox{\hrule width 0.4em height 0.3pt}}
%\setlength{\parindent}{0pt}

% title.tex is generated by make_latex; it includes static information
% such as the title and author, but also the version and date (taken
% from ../MNI.pm)
\input title


\begin{document}
\thispagestyle{empty}
\maketitle
\tableofcontents

\section{Introduction}

The MNI Perl Library is a motley collection of modules that greatly
facilitate writing the types of Perl programs commonly written for
medical-image analysis at the McConnell Brain Imaging Centre of the
Montreal Neurological Institute.  Quite simply, these programs are
glorified shell scripts: they act as front-ends to a series of other
programs (usually written in C, but potentially other Perl or shell
front-ends) that do the actual gruntwork of number-crunching and
image-processing.  The job of these ``glorified shell scripts'' is to
determine how exactly to run these varied sub-programs, cleverly
balancing the user's wishes, the need for correct results, and the
desire for efficiency.  (The delicacy of this balancing act is why
people generally give up on shell programming for these tasks in a big
hurry.)  Important requirements for such programs include:

\begin{itemize}
\item a flexible command-line option processor
\item careful checking of the surrounding environment to maximize the
  chances of child programs completing successfully
\item intelligent interaction with filenames and the filesystem
\item complete logging of all programs run and their arguments
\item consistent handling of crashed child programs
\item $\ldots$and the ability to turn all of these features off at will
\end{itemize}

The MNI Perl Library provides all of this and more.  This document is a
complete reference to all the modules in the library; it provides
exactly the same information you can get from the on-line manual pages.
A somewhat gentler introduction can be found in my document \emph{Tips
  for Perl Hackers}, also distributed with the library\footnote{Well,
  eventually it will be}.

\input inputs

\end{document}
