\documentclass{article}

\usepackage{times}

% Little macro to make keywords obvious.
\newcommand{\kw}[1]{{\tt \mbox{#1}}}

\title{MNI .obj file format}
\author{Robert D. Vincent \\
McConnell Brain Imaging Centre \\
Montreal Neurological Institute \\
Montreal, QC H3A 2B4 Canada \\~\\
}
\date{\today}
\begin{document}
\maketitle
% \newpage
\tableofcontents
\newpage

\section{Introduction}

The MNI .obj file format was designed as a simple storage format for
representations of geometric objects.  Overall, an
MNI .obj file may have either an ASCII and a binary representation.  In
ASCII files, space and newline characters serve as separators for data
fields, but have no other significance to the format.

All of the original design and implementation of this format was the
work of David MacDonald.  This document borrows extensively from David's
existing comments and documentation.

Also note that this specification was developed with the SGI GL and
OpenGL graphics programming systems in mind.  As a result much of the
terminology and format details reflect the capabilities and options
supported by these systems.

\section{Basics}

A .obj file consists of a simple list of object records which may be in
one of seven classes.  Each object record is introduced by a single
ASCII alphabetic character that determines the class of the object
record.  For binary records, the character is in lower case.  For ASCII
records, the character is in upper case.  The size of the record itself
is variable, but is typically determined by certain key fields that
define the number of vertices or points in the object.

\subsection{Basic types}

File data consists of five fundamental types: {\it char} (alphabetic
object class specifiers), {\it float}, {\it integer}, {\it string}, or
{\it boolean}.

Because the ASCII library is implemented using standard ``C'' functions
such as \kw{fscanf} and \kw{fprintf}, the textual representations of
numeric types are quite standard.  {\it Integer} values
consist of an optional sign ('+' or '-') followed by one or more decimal
digits. {\it Float} values will consist of an optional sign followed
by an integer part consisting of one or more decimal digits, followed by
an optional decimal point, followed by an optional fractional part
consisting of one or more decimal digits, and finally an optional
exponent specifier of the form 'e' or 'E' followed by an optional sign
and an integer exponent.  Either the integer part or the fractional part
must have a non-zero length.

In the case of binary files, there is no clear specification of the
format or ``endianness'' of the data.  It can be assumed that {\it integer}
values will be 32-bit, 2's complement integers, and that {\it float}
numbers will be 32-bit IEEE floating point values.

In ASCII files, a {\it string} is represented as a sequence of zero or more
characters surrounded by either single quotes, double quotes, or back
quotes.

In binary files, a {\it string} is represented as a 32-bit integer length
followed by the specified number of 8-bit bytes.

In ASCII files, {\it boolean} values are represented by the ASCII character
'f' or 'F' for FALSE, and 't' or 'T' for TRUE.

In binary files, {\it boolean} values are simply integer values.  Any non-zero
value is interpreted as representing logical TRUE, zero is interpreted
as FALSE.

\subsection{Compound types}

The MNI .obj format defines four compound types: {\it point}, 
{\it colour}, {\it vector}, and {\it surfprop}, a surface property record.

A {\it point} is defined by three floating point numbers corresponding to the
point's location along the X, Y, and Z axis respectively.  The range of
these values is arbitrary.

The second compound type defined by this format is the {\it colour} object.  A
colour is defined by four floating-point numbers in the interval 0..1,
representing red, blue, green, and alpha values.  The alpha value is
typically one.

A {\it vector} is similar to a point object in structure, consisting of three
values for the X, Y, and Z components of the vector.  Vectors are
assumed to be normalized unit vectors, therefore all values should be in
the range -1.0...+1.0.

A {\it surfprop} structure is composed of five floating point
parameters which define the appearance of a surface material. The
definition of these values was based largely on the SGI IRIS Graphics
Library and OpenGL specifications.  The five members of a surface
property structure are defined as follows:
\begin{itemize}
\item Ambient colour (A)

This value in the range 0...1 defines the intensity of the surface's
ambient colour, as a proportion of the surface's specified colour.

\item Diffuse reflectivity (D)

This value in the range 0...1 sets the diffuse reflectivity of the
object's surface, as a proportion of the surface's specified colour.

\item Specular reflectance (S)

This value in the range 0...1 specifies the specular reflectance of the
object surface.  Specular reflectance is assumed to be uniform over the
colour spectrum.

\item Specular scattering exponent (SE)

This value in the range 0...128 specifies the specular scattering
exponent of the material.  The higher the value, the smoother the
surface appearance and the more focused the specular highlight.

\item Transparency (T)

This value in the range 0...1 specifies the transparency of the surface.
This value may be ignored in many implementations, depending on the
graphics library in use and the capabilities of the graphics controller.

\end{itemize}

\subsection{Object classes}

The following eight object classes are defined by the MNI .obj file
format.

\begin{itemize}
\item 'L' (Lines) - A group of line segments.

\item 'M' (Marker) - A tag or marker point.

\item 'F' (Model) - A reference to an external .obj file.

\item 'X' (Pixels) - An 2-dimensional bitmap.

\item 'P' (Polygons) - A set of closed polygons.

\item 'Q' (Quadmesh) - A quadrilateral mesh.

\item 'T' (Text) - A text label.

\end{itemize}

The 'V' (Volume) character is reserved but not utilized.

\section{Object record details}

\subsection{Lines}

A lines object is used to represent a series of line segments.

\begin{itemize}

\item \kw{thickness} ({\it float})

This value specifies the overall thickness of the line segments.  It
must be within the range 0.001...20.0.

\item \kw{npoints} ({\it integer})

This specifies the total number of distinct vertices in the lines object

\item \kw{point\_array} (array of {\it point})

A list of the coordinates for each distinct vertex in the lines object.
Note that vertices may be reused if the \kw{end\_indices} and \kw{
indices} fields are set appropriately.

\item \kw{nitems} ({\it integer})

The total number of line segments

\item \kw{colour\_flag} ({\it integer})

A flag indicating the number of colours allocated in this object.  A
value of zero specifies that single colour applies to all line segments.
A value of one specifies that colours are specified on a per-line basis.
A value of two specifies that colours are specified on a per-vertex basis.

\item \kw{colour\_table} (array of {\it colour})

1The RGB colour values to be associated with the line segments.  The
length of this section may be either 1 (if \kw{colour\_flag} is 0),
\kw{nitems} (if \kw{colour\_flag} is 1) or \kw{npoints} (if \kw{
colour\_flag} is 2).

\item \kw{end\_indices} (array of {\it integer})

This is a list of length \kw{nitems} that specifies the index of the
last element in the \kw{indices} array for each group of line segments
in the lines object.

\item \kw{indices} (array of {\it integer})

The integer index into the \kw{point\_array} that specifies how each of the
vertices is assigned to each line segment.  The length of this array must
be equal to the greatest value in the \kw{end\_indices} array plus one.

\end{itemize}

\subsection{Marker}

A marker object is used to assign a special meaning to a position in the
object.  The marker can be associated with a specific patient,
anatomical structure, and text comment.

\begin{itemize}

\item \kw{type} ({\it integer}) 

A value of zero indicates this marker should be displayed as a box,
while a value of one indicates a sphere.

\item \kw{size} ({\it float})

The desired size of the marker.  The units and interpretation are not
clearly specified.

\item \kw{colour} ({\it colour})

The desired colour for the marker.

\item \kw{position} ({\it point})

The location of the maker.

\item \kw{structure\_id} ({\it integer})

An integer value associating this marker with a specific anatomical
structure.

\item \kw{patient\_id} ({\it integer})

An integer value associating this marker with a particular patient.

\item \kw{label} ({\it string})

A string to be displayed along with the marker.

\end{itemize}

\subsection{Model}

A model record is used to incorporate the contents of another .obj file
by reference.  Processing of model records is {\it not} automatically
handled by the library, it is up to the calling program to read the
external file.

\begin{itemize}

\item \kw{filename} ({\it string})

The path name of the file to incorporate in this object.

\end{itemize}

\subsection{Pixels}
A pixels record stores a two-dimensional bitmap object.

\begin{itemize}

\item \kw{pixel\_type} ({\it integer})

This field specifies the type of the \kw{pixel\_array} field.  The
pixel type can take a value of either 0, 1 or 2.  A value of zero
indicates that the pixel array values are eight bit color indices of
type {\it integer}.  A value of one indicates that pixel values are
sixteen bit color indices of type {\it integer}.  A value of two
indicates that pixel values are of RGB values of type {\it colour}.

\item \kw{x\_size} ({\it integer})

The extent of the bitmap in the horizontal direction.

\item \kw{y\_size} ({\it integer})

The extent of the bitmap in the vertical direction.

\item \kw{pixel\_array} (see \kw{pixel\_type} field)

The array of pixel values, with a total of $x\_size * y\_size$ elements.
Pixels are stored such that the upper left corner is the first pixel in
the array, and the lower right corner is the last pixel in the array.

\end{itemize}

\subsection{Polygons}

Polygon records may be in either compressed or uncompressed
format. Compressed format is reserved for sets of polygons with
tetrahedral topology.

The uncompressed format is as follows:

\begin{itemize}
\item \kw{surfprop} ({\it surfprop})

Surface properties for the polygons.

\item \kw{npoints} ({\it integer})

Number of distinct vertices in the aggregate polygon object.

\item \kw{point\_array} (array of {\it point})

List of distinct vertices that define this group of polygons.  Note that
vertices may be reused if the \kw{end\_indices} and \kw{indices}
fields are set appropriately.

\item \kw{normals} (array of {\it vector})

List of point normals for each point.

\item \kw{nitems} ({\it integer})

Number of polygons defined.

\item \kw{colour\_flag} ({\it integer})

A flag indicating the number of colours allocated in this object.  A
value of zero specifies that single colour applies to all line segments.
A value of one specifies that colours are specified on a per-item basis.
A value of two specifies that colours are specified on a per-vertex
basis.

\item \kw{colour\_table} (array of {\it colour})

The RGB colour values to be associated with the polygons.  The length of
this section may be either 1 (if \kw{colour\_flag} is 0), \kw{nitems}
(if \kw{colour\_flag} is 1) or \kw{npoints} (if \kw{colour\_flag} is
2).

\item \kw{end\_indices} (array of {\it integer})

This is a list of length \kw{nitems} that specifies the index of the
element in the indices list associated with each successive polygon.

\item \kw{indices} (array of {\it integer})

A list of integer indices into the \kw{point\_array} that specifies how
each of the vertices is assigned to each polygon.  The length of this
array must be equal to the greatest value in the \kw{end\_indices}
array plus one.

\end{itemize}

The compressed format is distinguised by the sign of the field after the
surface properties, which should always be negative.  Use of the
compressed format is reserved for polygons with tetrahedral topology.

\begin{itemize}
\item \kw{surfprop} ({\it surfprop})

Surface properties for the polygons.

\item -\kw{nitems} ({\it integer})

Number of polygons, multiplied by -1.

\item \kw{point\_array} (array of {\it point})

List of vertices that define this group of polygons.

\item \kw{colour\_flag} ({\it integer})

A flag indicating the number of colours allocated in this object.  A
value of zero specifies that single colour applies to all line segments.
A value of one specifies that colours are specified on a per-item basis.
A value of two specifies that colours are specified on a per-vertex basis.

\item \kw{colour\_table} (array of {\it colour})

The RGB colour values to be associated with the polygons.  The
length of this section may be either 1 (if \kw{colour\_flag} is 0), nitems
(if \kw{colour\_flag} is 1) or npoints (if \kw{colour\_flag} is 2).

\end{itemize}

\subsection{Quadmesh}

Quadrilateral mesh.

\begin{itemize}

\item \kw{surfprop} ({\it surfprop})

Surface properties for the mesh.

\item \kw{m} ({\it integer})

The number of rows in the mesh.

\item \kw{n} ({\it integer})

The number of columns in the mesh.

\item \kw{m\_closed} ({\it boolean})

A boolean specifying whether the rows have their extreme points joined
by an edge.

\item \kw{n\_closed} ({\it boolean})

A boolean specifying whether the columns have their extreme points
joined by an edge.

\item \kw{colour\_flag} ({\it integer})

A flag indicating the number of colours allocated in this object.  A
value of zero specifies that single colour applies to all line segments.
A value of one specifies that colours are specified on a per-item basis.
A value of two specifies that colours are specified on a per-vertex basis.

\item \kw{colour\_table} (array of {\it colour})

The RGB colour values to be associated with the mesh.  The
length of this section may be either 1 (if \kw{colour\_flag} is 0), 
$(m - 1) * (n - 1)$ (if \kw{colour\_flag} is 1) or $m * n$ 
(if \kw{colour\_flag} is 2).

\item \kw{point\_array} (array of {\it point})

A list of points, of total length $n * m$.

\item \kw{normals} (array of {\it vector})

A list of point normals, of total length $n * m$.

\end{itemize}

\subsection{Text}

A text object specifies a text label.

\begin{itemize}

\item \kw{font\_type} ({\it integer})

This integer flag is set to zero to specify a fixed-width font, or to
one for proportional spaced font.

\item \kw{text\_size} ({\it float})

The desired font size.

\item \kw{colour} ({\it colour})

The desired text colour.

\item \kw{point} ({\it point})

The 3D position at which to display the text.

\item \kw{text} ({\it string})

The text to display.

\end{itemize}

\section{Implementation}

The object file format is implemented by the ``BIC programmer's
library'' (BICPL).  The low-level I/O functions for basic types are
defined in the ``volume\_io'' library.  

The following BICPL functions are used to implement high-level
object file I/O:

\begin{verbatim}
Status input_graphics_file(char *filename, 
                           File_formats *format,
                           int *n_objects, 
                           object_struct ***object_list);

Status output_graphics_file(char *filename, 
                            File_formats format,
                            int n_objects, 
                            object_struct *object_list[]);
\end{verbatim}

\section{Related file formats}

\subsection{Transform files}

MNI transform files are ASCII text files that define a transformation
between two coordinate systems.

\subsection{Tag point files}

MNI tag point files are ASCII text files that define a series of points
within a volume, or a correspondence between points in two separate
volumes.

The format is implemented and documented in the {\tt libminc} and {\tt
 volume\_io} libraries.

\end{document}
